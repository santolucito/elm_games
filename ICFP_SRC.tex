%-----------------------------------------------------------------------------
%
%               Template for sigplanconf LaTeX Class
%
% Name:         sigplanconf-template.tex
%
% Purpose:      A template for sigplanconf.cls, which is a LaTeX 2e class
%               file for SIGPLAN conference proceedings.
%
% Guide:        Refer to "Author's Guide to the ACM SIGPLAN Class,"
%               sigplanconf-guide.pdf
%
% Author:       Paul C. Anagnostopoulos
%               Windfall Software
%               978 371-2316
%               paul@windfall.com
%
% Created:      15 February 2005
%
%-----------------------------------------------------------------------------


\documentclass{sigplanconf}

% The following \documentclass options may be useful:

% preprint      Remove this option only once the paper is in final form.
% 10pt          To set in 10-point type instead of 9-point.
% 11pt          To set in 11-point type instead of 9-point.
% authoryear    To obtain author/year citation style instead of numeric.

\usepackage{amsmath}


\begin{document}

\special{papersize=8.5in,11in}
\setlength{\pdfpageheight}{\paperheight}
\setlength{\pdfpagewidth}{\paperwidth}

\conferenceinfo{CONF 'yy}{Month d--d, 20yy, City, ST, Country} 
\copyrightyear{20yy} 
\copyrightdata{978-1-nnnn-nnnn-n/yy/mm} 
\doi{nnnnnnn.nnnnnnn}

% Uncomment one of the following two, if you are not going for the 
% traditional copyright transfer agreement.

%\exclusivelicense                % ACM gets exclusive license to publish, 
                                  % you retain copyright

%\permissiontopublish             % ACM gets nonexclusive license to publish
                                  % (paid open-access papers, 
                                  % short abstracts)

\titlebanner{banner above paper title}        % These are ignored unless
\preprintfooter{short description of paper}   % 'preprint' option specified.

\title{Javascript as an Intermediate Language for FRP}
\subtitle{Subtitle Text, if any}

\authorinfo{Name1}
           {Affiliation1}
           {Email1}

\maketitle

%\begin{abstract}
%\end{abstract}

\category{CR-number}{subcategory}{third-level}

% general terms are not compulsory anymore, 
% you may leave them out
\terms
term1, term2

\keywords
keyword1, keyword2

\section{Introduction}


%probelm
Functional Reactive Programming (FRP) is an exciting and popular way to make interactive applications in a functional style.
However, creating production level applications with FRP remains prohibitivly difficult, especially for mobile platforms.
While many functional programmers have the programming experience to write FRP applications, technicial issues have kept this circle to only the most dedicated FRP researchers.
Using Javascript as an intermediate language for FRP code eliminates most of these technical hurdles and makes publishing applications written with FRP an option for more users.


%motivation

FRP exists as libraries in Haskell, the dedicated FRP language Elm, and Sodium is a project to port FRP to more language like Java and C.
However, FRP often feels most natural in functional language like Haskell and Elm.
Building a system that easily 
works very well in high-level functioal languages, but Android is in Java.
We want to write FRP programs for Android.
Many FRP libraries exist in Haskell, and Elm is a dedicated FRP language that is very similar to Haskell.

Apps can be developed and maintained much faster using FRP.
prototypes can be generated very quickly.

The development of FRP and functional languages in general relies on adoption by industry.
Simplifying the work flow to bring FRP code to market will encourage future work in FRP.

%related
Ivan Perez from Keera Stuios has cross-compiled haskell to andoird and published a game.
His approach is complicatd and closed-source at the moment.
Java has an FRP library, but Java is not well suited for FRP.

%soltn
Compile FRP program to an intermediate language, then compile to Android.
The intermediate language 
By using ghcjs, we can compile our Haskell FRP program to js.
Then using Phonegap, we can comile that generated js to an Android application.
This is unique because it removes all the cross platform difficulties.

%APIs
There are two kinds of APIs for moblie development, software and hardware.
Software APIs (e.g. google maps) would require a .jar file loaded to the android project - now we just use the javascript API for that library.
Hardware APIs (e.g. acceleration data) use device specific APIs.
Some javascript APIs exist for functionality most devices have, like aceleraometers.

If native code is necessary, Phonegap provides \texit{plugins} as a foregin function interface (FFI) to javscript.
Plugin code can be called via the FFI in javscript, which is in turn called through the source languages FFI.
Both ghcjs and Elm provide a javascript FFI. 

Development may also be split between any combination of the FRP language, javascript, and the target language.


%Results
We developed a proof of concept app using Elm and have provided the source code and app files at \www{github.com/santolucito/elm_games}.
This app uses FRP to read the touch interface native to the mobile hardware and control a game.
The app also uses a plugin to show ads, mixing javascript code with Elm.
It can run at ?? frames per second on an Nexus 5 running Android verison X.XX.
pass in (fps 200) as the time signal to a display of "toString (1000/t)" to meausre actual FPS - capped at 200 on laptop. 
 
\appendix
\section{Appendix Title}

This is the text of the appendix, if you need one.

\acks

Acknowledgments, if needed.

% We recommend abbrvnat bibliography style.

\bibliographystyle{abbrvnat}

% The bibliography should be embedded for final submission.

\begin{thebibliography}{}
\softraggedright

\bibitem[Smith et~al.(2009)Smith, Jones]{smith02}
P. Q. Smith, and X. Y. Jones. ...reference text...

\end{thebibliography}


\end{document}

%                       Revision History
%                       -------- -------
%  Date         Person  Ver.    Change
%  ----         ------  ----    ------

%  2013.06.29   TU      0.1--4  comments on permission/copyright notices

